%-----------------------------------------------------------------------------%
\chapter{\babTiga}
\label{bab:3}
%-----------------------------------------------------------------------------%
Bab ini menjelaskan tentang hal-hal \f{advanced} dalam \latex. Hal ini mencakup bagaimana cara menulis persamaan matematis di \latex, menambahkan daftar isi, catatan, PDF, menambahkan kode, bahkan menambahkan perintah baru.

\todo{Sejatinya bab ini digunakan untuk membahas inti dari penelitian Anda. Sesuaikan saja dengan kebutuhkan Anda: misalkan bab tiga Anda adalah penjelasan terkait desain sistem.}


%-----------------------------------------------------------------------------%
\section{Membuat Persamaan Matematis}
\label{sec:mathEqu}
%-----------------------------------------------------------------------------%
Di \latex, kita dapat membuat persamaan matematis baik yang terdiri dari satu persamaan maupun lebih dari satu persamaan. Anda bisa mencoba mengikuti dan memahami contoh kode yang ada di \f{template} ini untuk kebutuhan tugas akhir Anda. Menggunakan \latex~juga perlu latihan dan lihai memahami dokumentasi.

%-----------------------------------------------------------------------------%
\subsection{Satu Persamaan}
\label{sec:oneEqu}
%-----------------------------------------------------------------------------%

\noindent \begin{align}\label{equ:garis}
	\cfrac{y - y_{1}}{y_{2} - y_{1}} =
	\cfrac{x - x_{1}}{x_{2} - x_{1}}
\end{align}

\equ~\ref{equ:garis} diatas adalah persamaan garis.
\equ~\ref{equ:garis} dan \ref{equ:bola} sama-sama dibuat dengan perintah \code{\bslash{}align}.
Perintah ini juga dapat digunakan untuk menulis lebih dari satu persamaan.

\noindent \begin{align}\label{equ:bola}
	\underbrace{|\overline{ab}|}_{\text{pada bola $|\overline{ab}| = r$}}
	= \sqrt[2]{(x_{b} - x_{a})^{2} + (y_{b} - y_{a})^{2} +
		\vert\vert(z_{b} - z_{a})^{2}}
\end{align}
