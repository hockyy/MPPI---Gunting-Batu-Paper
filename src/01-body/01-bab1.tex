%-----------------------------------------------------------------------------%
\chapter{\babSatu}
\label{bab:1}
%-----------------------------------------------------------------------------%
Bab ini membahas latar belakang yang menjabarkan perkembangan pemungutan suara elektronik, bagaimana penerapannya secara global maupun dalam organisasi-organisasi dalam lingkup tertentu, serta kondisi pemilihan raya (pemira) yang dilakukan oleh Ikatan Keluarga Mahasiswa Universitas Indonesia beberapa tahun terakhir. Selanjutnya bab ini juga membahas mengenai rumusan masalah yang diperoleh dari latar belakang yang disampaikan, tujuan dan manfaat penelitian, serta sistematika penulisan untuk penelitian yang akan penulis lakukan.


%-----------------------------------------------------------------------------%
\section{Latar Belakang}
%-----------------------------------------------------------------------------%
\todo{Faishol kerjain latar belakang, tujuan, manfaat, batasan kita fixin lagi ↓↓↓↓↓ yang di bawah masih kurang relevan}

Pemungutan suara merupakan proses pengumpulan pendapat berdasarkan suara mayoritas untuk menentukan pilihan bersama yang menjadi alternatif bagi musyawarah. Digunakannya media elektronik dalam pemungutan suara mulai diterima masyarakat karena kemudahannya dalam aspek aksesibilitas. Namun, sebagian metode pemungutan suara elektronik saat ini masih memiliki kekurangan dari segi privasi, keamanan dan akuntabilitas \citep{7571928}. 

Pemungutan suara elektronik atau \textit{e-voting}, baik melalui internet atau hanya memanfaatkan teknologi sebagai media verifikasi dan penyimpanan hasil pemungutan suara sudah pernah diimplementasikan dalam pemilihan umum nasional di berbagai negara \citep{7001135}, dan justru tidak jarang diimplementasikan oleh berbagai negara berkembang \citep{reviewevote}. Salah satunya adalah Nigeria, berdasarkan laporan pasca-pemilu, penggunaan \textit{e-voting} dipercaya lebih kredibel dan mengurangi risiko terjadinya manipulasi \citep{nigeria}. Lain halnya dengan Brazil, terdapat beberapa isu terkait keterbukaan dan transparansi pihak pemerintah yang beredar di masyarakat \citep{avgerou}. Fakta bahwa jutaan masyarakat Brazil masih belum mampu mengakses teknologi internet untuk pemilihan umum yang bersifat global, juga menjadi salah satu masalah bagi keputusan pemerintah di sana \citep{mci/Filho2008}.

Selain itu terdapat beberapa isu lain, baik minor maupun fatal yang muncul dalam pemilihan-pemilihan umum yang pernah dilaksanakan di berbagai negara. Salah satunya ialah transparansi kepada masyarakat, keamanan terhadap manipulasi, dan kurangnya literasi penduduk terhadap aturan dan teknologi canggih yang digunakan dalam sistem pemungutan suara elektronik ini \citep{7001136}. Hal ini membuka berbagai peluang bagi para ilmuwan untuk terus mencari mekanisme-mekanisme \textit{e-voting} yang lebih baik.

Dari berbagai metode pemungutan suara elektronik yang ada, sebagian besar basis data masih terpusat pada pemerintah. Hal ini merupakan salah satu faktor yang dapat menyebabkan berkurangnya tingkat kepercayaan masyarakat, yang berbeda dengan \textit{voting} secara langsung dan transparan apabila menggunakan surat suara fisik \citep{8651451}. Terdapat beberapa solusi yang ditawarkan oleh para peneliti, salah satunya ialah teknologi \textit{blockchain} yang terdesentralisasi.

%-----------------------------------------------------------------------------%
\section{Rumusan Masalah}
%-----------------------------------------------------------------------------%
Berdasarkan latar belakang yang dipaparkan sebelumnya, rumusan masalah yang diangkat dalam penelitian ini adalah untuk mengetahui faktor apa saja yang dapat meningkatkan preferensi pemilih dalam menggunakan sistem pemungutan suara elektronik dibandingkan pemungutan suara konvensional.

%-----------------------------------------------------------------------------%
\section{Tujuan Penelitian}
%-----------------------------------------------------------------------------%
Karya ini bertujuan untuk memanfaatkan perkembangan gagasan teknologi informasi dengan metode-metode kriptografi dalam pemungutan suara elektronik, serta dapat memaparkan faktor-faktor yang dapat meningkatkan taraf partisipasi pemungutan suara elektronik dibandingkan metode pemungutan suara konvensional, sehingga hasilnya bisa diterapkan secara langsung dan menjadi acuan bagi penelitian selanjutnya.

%-----------------------------------------------------------------------------%
\section{Manfaat Penelitian}
%-----------------------------------------------------------------------------%
Karya ini diharapkan dapat meningkatkan pengetahuan masyarakat Indonesia akan keamanan dan cara kerja pemungutan suara elektronik secara umum, serta memberikan gambaran gagasan arsitektur yang jelas, dan dapat menjadi referensi pembuatan sistem atau penelitian selanjutnya.

%-----------------------------------------------------------------------------%
\section{Batasan Penelitian}
%-----------------------------------------------------------------------------%
Penelitian ini terbatas dalam konteks studi kasus pemilihan raya Fakultas Ilmu Komputer Universitas Indonesia.

%-----------------------------------------------------------------------------%
\section{Sistematika Penulisan}
%-----------------------------------------------------------------------------%
Proposal karya ilmiah ini terdiri dari empat bab, antara lain (1) pendahuluan, (2) tinjauan pustaka, (3) metodologi penelitian, dan (4) kesimpulan. Bab pendahuluan membahas latar belakang penelitian dan beberapa rumusan pertanyaan penelitian yang diajukan berdasarkan masalah yang dijabarkan. Bagian tinjauan pustaka menjelaskan lebih lanjut literatur-literatur yang menjadi referensi dalam penelitian ini. Bab metodologi penelitian membahas mengenai tahapan dan instrumen yang digunakan dalam penelitian ini.