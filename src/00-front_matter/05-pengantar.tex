%-----------------------------------------------------------------------------%
\chapter*{\kataPengantar}
%-----------------------------------------------------------------------------%

Puji syukur kami panjatkan ke hadirat Tuhan Yang Maha Esa, karena rahmat dan anugerah-Nya, penulis dapat menyelesaikan proposal penelitian yang berjudul “\f{\judul}” yang menjadi salah satu indikator evaluasi mata kuliah Metodologi Penelitian dan Penulisan Ilmiah.

Karya tulis ini dapat disusun dengan baik dan tepat waktu, atas bantuan dan dukungan dari berbagai pihak, dan tentu saja dengan hikmat dan kehendak-Nya. Karya tulis ini terinspirasi atas ide pemungutan suara elektronik yang dapat diselenggarakan untuk mengurangi kontak fisik dalam situasi pandemi Covid-19. Ada pula gagasan sistem yang disampaikan merupakan hasil sintesis karya-karya dengan topik yang sama sebelumnya dan buah pemikiran dari penulis. Penulis juga ingin berterima kasih kepada pihak-pihak lain, khususnya kepada:

\begin{itemize}
  \setlength\itemsep{0em}
    \item orang tua para penulis yang turut serta mendukung dalam menjalankan proses perkuliahan sehari-hari sembari menyelesaikan karya ilmiah ini;
    \item Ibu Dr.Eng. Laksmita Rahadianti, S.Kom., M.Sc. dan Ibu Annisa Monicha Sari, S.Kom., M.Kom. selaku dosen pembimbing yang menjadi pembimbing dalam mengemban mata kuliah ini;
    \item asisten dosen pembimbing, Raihan Rizqi Muhtadiin yang memberikan inspirasi, koreksi, serta tanggapan untuk karya ilmiah ini; dan
    \item rekan-rekan kakak tingkat serta teman-teman penulis yang memberikan saran dalam penyusunan karya ilmiah ini.
\end{itemize}

Penulis juga menyadari bahwa masih terdapat kesalahan dan kekurangan dalam penulisan karya ilmiah ini. Penulis berharap karya tulis ini dapat memberikan manfaat dan inspirasi untuk pengembangan dan peradaban ilmu pengetahuan teknologi dan informatika dunia, terutama bangsa Indonesia.

\vspace*{0.1cm}
\begin{flushright}
Depok, \tanggalSiapSidang\\[0.1cm]
\vspace*{1cm}
\penulis

\end{flushright}
