%-----------------------------------------------------------------------------%
% Informasi Mengenai Dokumen
%-----------------------------------------------------------------------------%
%
% Judul laporan. 
\var{\judul}{Analisis Faktor Peningkatan Partisipasi Pemungutan Suara Elektronik dalam Studi Kasus Pemilihan Raya Fasilkom UI}
% 
% Tulis kembali judul laporan, kali ini akan diubah menjadi huruf kapital
\Var{\Judul}{Analisis Faktor Peningkatan Partisipasi Pemungutan Suara Elektronik dalam Studi Kasus Pemilihan Raya Fasilkom UI}
%
% Tulis kembali judul laporan namun dengan bahasa Ingris
\var{\judulInggris}{Your Scientific Publication Title}

% 
% Tipe laporan, dapat berisi Skripsi, Tugas Akhir, Thesis, atau Disertasi
\var{\type}{Proposal Mini Riset}
% 
% Tulis kembali tipe laporan, kali ini akan diubah menjadi huruf kapital
\Var{\Type}{Proposal Mini Riset}
%
% Tulis nama penulis 
\var{\penulis}{Tim Penulis}
% 
% Tulis kembali nama penulis, kali ini akan diubah menjadi huruf kapital
\Var{\Penulis}{Tim Penulis}
%
% Tulis NPM penulis
\var{\npm}{Nomor Anda}
%
% Tuliskan Fakultas dimana penulis berada
\Var{\Fakultas}{Ilmu Komputer}
\var{\fakultas}{Ilmu Komputer}
% 
% Tuliskan Program Studi yang diambil penulis
\Var{\Program}{ILMU KOMPUTER}
\var{\program}{Ilmu Komputer}
% 
% Tuliskan tahun publikasi laporan
\Var{\bulanTahun}{April 2022}
%
% Tuliskan gelar yang akan diperoleh dengan menyerahkan laporan ini
\var{\gelar}{Gelar Jurusan Anda}
%
% Tuliskan tanggal pengesahan laporan, waktu dimana laporan diserahkan ke
% penguji/sekretariat
\var{\tanggalSiapSidang}{Tanggal Bulan Tahun}
%
% Tuliskan tanggal keputusan sidang dikeluarkan dan penulis dinyatakan
% lulus/tidak lulus
\var{\tanggalLulus}{Tanggal Bulan Tahun}
% Tuliskan tanggal pengesahan laporan final, waktu dimana laporan
% diserahkan ke perpustakaan
\var{\tanggalFinal}{Tanggal Bulan Tahun}
%
% Tuliskan pembimbing
\var{\pembimbingSatu}{Pembimbing Pertama Anda}
\var{\pembimbingDua}{Pembimbing Kedua Anda}
%
% Tuliskan penguji
\var{\pengujiSatu}{Penguji Pertama Anda}
\var{\pengujiDua}{Penguji Kedua Anda}

%-----------------------------------------------------------------------------%
% Judul Setiap Bab
%-----------------------------------------------------------------------------%
%
% Berikut ada judul-judul setiap bab.
% Silahkan diubah sesuai dengan kebutuhan.
%
\Var{\kataPengantar}{Kata Pengantar}
\Var{\babSatu}{Pendahuluan}
\Var{\babDua}{Landasan Teori}
\Var{\babTiga}{Notasi Matematik}
\Var{\babEmpat}{Struktur Berkas}
\Var{\babLima}{Perintah dalam uithesis.sty}
\Var{\babEnam}{Bab Enam}
\Var{\kesimpulan}{Kesimpulan dan Saran}
